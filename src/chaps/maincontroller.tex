\section{Main Controller Design}
\subsection{Inputs}
\begin{itemize}
  \item \FF{1\ldots3}
  \item \FF{456}
\end{itemize}
These inputs notify the Main Controller when a flow is inished. \textbf{Note:}
\FF{456} is collected as the output of a secondary controller chip (called
``Minor Controller'') due to a lack of inputs on the Main Controller. \FF{456}
is defined as
\begin{align}\FF{456} &= \FF{4} + \FF{5} + \FF{6}\end{align}
This works because the next flow following flows 4 through 6 will always be flow
1.

\begin{itemize}
  \item \AR
  \item \B
  \item \CR
  \item \D
\end{itemize}
These inputs are sensor inputs from loop sensors in the road. It is assumed that
the input has been amplified and buffered\footnote{by someone in ELEC170} to
present a 5V logic \HIGH when a car is over the sensor.
\begin{itemize}
  \item \nReset
\end{itemize}
A synchronous reset used to defer back to the default flow upon initialization
of the system or error.
\begin{itemize}
  \item \nS
\end{itemize}
A logical \texttt{OR} of sensor inputs \AR, \B, \CR, \D.

\subsection{Outputs}
\begin{itemize}
  \item EN{1\ldots 6}
\end{itemize}
Enable flows: Thes go \HIGH to switch a flow on.

\begin{itemize}
  \item \nS
\end{itemize}
The sensor \nS was provided to be passed on to flow 1 since flow
one is the only flow that is in any way privy to the actual movements of the
sensors. Simply put, as per specifications, flow 1 is to remain in operation
until any one of the sensors goes \HIGH. This enables it to be the default flow
which continues to relieve traffic in the most busy direction if no other
traffic is detected. Thus, the equation for S is quite simple:
\begin{align}\texttt{S.D = \AR + \B + \CR + \D}\end{align}
The reason why we have \FF{456} as a single input is simple: we ran out of
inputs to the Main Controller chip. Luckily, all three of these flows (4, 5 and
6) lead directly back to flow 1 so we could take this short cut. Following
sections will show and explain the equations used on the Main Controller chip.