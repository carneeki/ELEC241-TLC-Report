\section{Flow 1 Chip}
The flow 1 chip is unique in that it is the only chip (at least partially)
privy to the movements of the sensors. It is also the only chip with a looping
state, and it is the longest flow. Thus its implementation is a bit more
complex. It is a modified MOD 20 counter with a looping state built in. 

\subsection{Inputs}
\begin{itemize}
  \item \nS
  \item \EN{}
\end{itemize}

\subsection{Outputs}
\begin{itemize}
  \item \FF
  \item \nGr
  \item \nAm
  \item \nRed (lights / light \emph{requests})
  \item \Q{[0\ldots4]} counter variables
\end{itemize}

Table \ref{tab:F1sm} on page \pageref{tab:F1sm} shows the state machine logic.

From table \ref{tab:F1sm} we were able to derive the equations for all
outputs using intermediary Karnaugh maps. The equations have been listed in the
appendix, listing \ref{lst:F1} on page \pageref{lst:F1}.

\textbf{Note:} for reasons that were explained earlier the equations for \nGr,
\nAm, \nRed(Green, Amber and Red) have no \texttt{.D} attached to them so as to
immediately forward on these light requests to flows 5 and 6. Also note that all
green equations in this whole project on every term have a term resembling
\texttt{/\nAm*\EN*(---)} in them, the \texttt{/\nAm} is required so that these
counters variables could be reduced by 1 to fit within three bits, this means
that for a 7 second cycle we only need a mod 8 counter which only requires
3 bits, but since the control chip takes one cycle to deactivate the \EN for
that flow, the green would switch back on as it resets to state 0, thus the
\texttt{/\nAm} input takes care of that single state while it is waiting for
the controller to switch it off. 